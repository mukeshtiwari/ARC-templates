%! program = pdflatex

\documentclass[12pt,a4paper]{article} 
\usepackage{arc-dp}
\usepackage{amsfonts}
\usepackage{amsmath}
\usepackage{graphicx}
\usepackage{times}
\usepackage{setspace}
\usepackage{fancyhdr}
\usepackage{color}
\usepackage[normalem]{ulem}
\usepackage{subfig}
\usepackage{float}
\usepackage{caption}
\usepackage{array}
\usepackage{pgfgantt}
\usepackage{url}
\usepackage{enumitem}
\usepackage{subfig}
\usepackage{pgfgantt}
\usepackage{wrapfig}
\usepackage{enumitem}

\usepackage{booktabs} % for spacing tables
\usepackage{tabularx} % auto table sizing
\usepackage{multirow} % table multirow

%\usepackage{epsf}
%\usepackage{fancyheadings}
%\usepackage{subfigure}
%\usepackage{pst-gantt}
%\usepackage{tweaklist}

\newcolumntype{L}[1]{>{\raggedright\let\newline\\\arraybackslash\hspace{0pt}}m{#1}}
\newcolumntype{C}[1]{>{\centering\let\newline\\\arraybackslash\hspace{0pt}}m{#1}}
\newcolumntype{R}[1]{>{\raggedleft\let\newline\\\arraybackslash\hspace{0pt}}m{#1}}

\let\OLDthebibliography\thebibliography
\renewcommand\thebibliography[1]{
  \OLDthebibliography{#1}
  \setlength{\parskip}{1pt}
  \setlength{\itemsep}{1pt plus 0.3ex}
}


%\renewcommand{\enumhook}{\setlength{\topsep}{0pt}%
 % \setlength{\itemsep}{-2mm}}
%\renewcommand{\itemhook}{\setlength{\topsep}{0pt}%
%  \setlength{\itemsep}{-2mm}}
  %%%%%UNCOMMENT THE NEXT COMMAND IF NEEDED
%\renewcommand{\descripthook}{\setlength{\topsep}{0pt}%
%  \setlength{\itemsep}{-2mm}}

%\pagestyle{fancy}

%\input{psfig.sty}
\newcommand{\todo}[1]{\textcolor{red}{#1}}
\newcommand{\rules}[1]{\textcolor{blue}{#1}}
\newcommand{\pset}{ {\rm P} \! \! \! {\rm P} }
\date{}
%\include{psfig}
\remove{
\topmargin -15mm
\headheight 0pt
\headsep 0pt
\textheight 285mm
\oddsidemargin -15mm
\evensidemargin -15mm
\textwidth 190mm
\columnsep 10mm
\marginparwidth 0pt
\marginparsep 0pt
}

\usepackage[top=0.5cm, bottom=0.5cm, left=0.5cm, right=0.5cm]{geometry}
\parindent=4mm
\parskip=0.2mm

%\usepackage{geometry} % see geometry.pdf on how to lay out the page. There's lots.
%\geometry{a4paper} % or letter or a5paper or ... etc
% \geometry{landscape} % rotated page geometry


%\linespread{1.5}

\newcommand*{\TitleFont}{%
      \usefont{\encodingdefault}{\rmdefault}{b}{n}%
      \fontsize{12}{12}%
      \selectfont}

\title{The Title of your ARC DP Proposal}
%\author{}
\date{} % delete this line to display the current date

%%% BEGIN DOCUMENT
\begin{document}
\rmfamily
\date{}


\noindent \textbf{PROJECT TITLE: }\\ \noindent Privacy-preserving computing for 
social good, underpinned by mathematical proof of correctness.

%Assisted Query Formulation for Cheaper, Faster & Unbiased Systematic Review


%Reducing Time and Cost for the Creation of Systematic Reviews through (Semi)-Automated Query Formulation\\ 
%\noindent (Semi-) Automatic Assisted Query Formulation for the Creation of Systematic Reviews


\subsection*{Application Summary}
Computer programs (software) are governing almost every aspect of Australian society. 
On one hand, they are making life convenient and efficient, on the other they 
are also providing opportunities to cybercriminals to steal the sensitive data of 
businesses and individuals in Australia. This project aims to address 
data breaches by using the recent advancement in cryptography. More precisely, 
it has two goals: (i) create novel privacy-preserving methods, particularly
in electronic-voting and electronic-health, using cutting-edge cryptographic techniques 
and (ii) use mathematics (constructive logic) to implement and prove these methods 
are secure and trustworthy. Expected outcomes of this project includes new knowledge 
to build trustworthy systems 

This should provide significant societal benefits by avoiding the data breaches. 




We intend to apply our findings 
in other domains, electronic-governance, electronic-payment, and electronic-identity  
and thereby significantly reducing the impact of data-breaches on Australian society. 



\subsection*{Investigator/Capability}
My research touches the lives of common people and solves
real-world problems that matter to democracies and common people. During my 7 years research career 
--ranging from Australian National University, University of Melbourne, University of Cambridge, 
and University of Oxford-- I have produced 9 mathematically proven correct 
software for public good, which includes software to count Australian Senate election ballots,
mix-network used in Switzerland election, software for both: plaintext ballot Schulze Method and 
encrypted ballot Schulze method, software to process sensitive data in trusted 
computing (Intel SGX and ARM Trustzone), software to model cryptographic protocols, 
software to verify elections conducted by International Association for Cryptologic Research, etc. 

In this section, I briefly explain my papers and how they solve a problem pertaining to social good.
My paper (i) \textbf{Assume but Verify: Deductive Verification of Leaked Information in Concurrent Applications} \cite{murrayassume},
accepted in ACM CCS 2023, develops a theory
for processing sensitive data --ethnic origin, political opinions,
health-related data, and  biometric data-- of common
people in secure enclave, e.g., Intel SGX, Arm TrustZone, etc. Moreover,
we demonstrate the usability of our method by developing non-trivial case studies that handles
sensitive data accompanied by the machine-checked mathematical proofs that none of
them have unintended side-channel data leakage;
(ii) \textbf{Machine-checking Multi-Round Proofs of Shuffle: Terelius-Wikstrom and Bayer-Groth} \cite{287095},
published in USENIX Security 2023, mathematically establishes a critical piece of
code in the SwissPost voting software --used in legally binding
elections in Switzerland-- is correct (and debunks a decade old myth of the cryptographic
community that Terelius-Wikstrom method is zero-knowledge proof. We have formally
proved in the Coq theorem prover that it is a zero-knowledge argument and not
a zero-knowledge proof);
(iii) \textbf{Verifiable Homomorphic Tallying for the Schulze Vote Counting Scheme} \cite{10.1007/978-3-030-41600-3_4}, published
in VSTTE 2019, not only developes a publicly verifiable method to count encrypted ballots for
a complex voting method but it
is also proven correct in the Coq theorem prover to ensure that there is no gap between
the pen-and-paper proof and the actual implementation;  (iv) \textbf{Verified Verifiers for
Verifying Elections} \cite{10.1145/3319535.3354247}, published in ACM CCS 2019, develops a mathematically proven correct tool
in the Coq theorem prover to verify the elections conducted by
the International Association for Cryptologic Research. We have used
our tools to verify the integrity of IACR elections; (v) \textbf{Modular Formalisation and
Verification of STV Algorithms} \cite{10.1007/978-3-030-00419-4_4}, published in E-Vote 2018, develops a mathematically proven
correct tool in the Coq theorem prover. We have used this tool to verify
the results of Australian Senate election;
(vi) \textbf{Verifpal: Cryptographic Protocol Analysis for the Real World} \cite{10.1007/978-3-030-65277-7_8}, published in
INDOCRYPT 2020, develops a tool that can used to model real work cryptographic protocol, and
Verifpal has been used by many researchers to model security and privacy aspect of
digital contact tracing during COVID; (vii) \textbf{Schulze Voting as Evidence Carrying Computation} \cite{10.1007/978-3-319-66107-0_26}, 
published in ITP 2017, develops the idea of proof-carrying computation for 
the Schulze Method, one of the widely used method over the Internet, etc. 
At the University of Cambridge, I developed a mathematically proven correct tool in 
the Coq theorem prover that can be used by researchers to model networking-protocols in the abstract
setting of semirings (and we are in the process of submitting our paper in CAV 2024).
At the University of Oxford, I am exploring the avenues to bridge the gap between a security protocol
(formal communication model) with its implementation using session types, and our
goal is to produce a more realistic distributed executable model of the security protocol.
Currently, as a first step, I am focussing on Signal app
where my goal is to prove (or disprove) that its Java implementation follows the
communication model described in the Signal's documentation.
In a nutshell, all my research so far has an impact on the lives on common people and
researchers and my DECRA proposal is a step towards continuing my research 
for the social good. 


\subsection*{PROJECT QUALITY AND INNOVATION}



\subsection*{BENEFIT}

\subsection*{Feasibility}


%\vspace{4pt}
\renewcommand{\refname}{\normalfont\selectfont\TitleFont REFERENCES} 
%\vspace{-14pt}
\begingroup
    \fontsize{10pt}{10pt}\selectfont
\bibliographystyle{abbrv}
\bibliography{references.bib}

%\todo{\subsection*{Changes to be made}}
%\todo{
%\begin{enumerate}
%\item bring clear aims, objectives and methods in the first page
%\item clarify how the method is unique, why cannot be done by Google or others. Clarify why it leads to a solution
%\item reduce the quantity of text
%\item add images: 1. example of good query and results vs example of bad query and result; 2. study methodology/phases
%\end{enumerate}}


\endgroup


\end{document}


